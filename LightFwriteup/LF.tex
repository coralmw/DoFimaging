\documentclass[11pt]{amsart}
\usepackage{geometry}                % See geometry.pdf to learn the layout options. There are lots.
\geometry{a4paper}                   % ... or a4paper or a5paper or ... 
%\geometry{landscape}                % Activate for for rotated page geometry
%\usepackage[parfill]{parskip}    % Activate to begin paragraphs with an empty line rather than an indent
\usepackage{graphicx}
\usepackage{amssymb}
\usepackage{epstopdf}
\usepackage{listings}
\lstset{language=Python}
\DeclareGraphicsRule{.tif}{png}{.png}{`convert #1 `dirname #1`/`basename #1 .tif`.png}

\title{Light Field Photography on no Budget}
\author{T GM Parks}
%\date{}                                           % Activate to display a given date or no date

\begin{document}
\begin{abstract}
\end{abstract}

\maketitle
%\section{}
%\subsection{}

\section{Introduction}
Light field photography involves using a spatially distributed array of cameras to capture a single scene in order to capture the color and direction of light t each pixel location. This can be used to provide higher quality imaging and novel post processing tools. Several high profile papers have been published from Stanford's research lab, and a spinoff company called Lytro has been formed selling consumer cameras based upon this technology.

Even so, all existing light field cameras have been large and expensive. This paper covers the development of a Raspberry Pi based computational camera in order to bring Light Field imaging to a wider audience. 

\section{hardware}

In order to capture and process the image, a Rasberry Pi was wased due to its wide range of interfaces, aavailability of a camera and price. The raspberry Pi camera is a 5MP cellphone image sensor connectedd to the RPi via CSI. The camera is physically mounted to a linear driven servo controled platform, and powered by a EasyDriver board connected to the RPi to move the ccamera latrally.

One effect of caturing the light feild is post-processing control of the focus point.


\end{document}  